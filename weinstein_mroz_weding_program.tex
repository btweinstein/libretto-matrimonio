%
\documentclass[12pt,twoside]{article}

\usepackage[paperwidth=5.8in, paperheight=8.3in]{geometry}
\setlength{\hoffset}{-0.60in} % this gives a .30in margin
\setlength{\textwidth}{5in}
\setlength{\voffset}{0in}
\setlength{\textheight}{6.5in} % 1in margin at top and bottom; default text height is 430pt
\setlength{\marginparwidth}{0in}
\setlength{\oddsidemargin}{0in}
\setlength{\evensidemargin}{0in}
\usepackage{parskip}
\usepackage{layout}

\usepackage{calligra}
\pagenumbering{gobble}

\usepackage[usenames,dvipsnames]{color}
\usepackage{fontspec}
%\usepackage{dashrule}
\usepackage{tabularx}
\usepackage{pbox}
\usepackage{tabu}
\usepackage{multicol}

%\usepackage[garamond]{mathdesign}
%\usepackage[T1]{fontenc}

\usepackage[T1]{fontenc}
\usepackage{garamondx}
\usepackage[garamondx,cmbraces]{newtxmath}

% To make a booklet
%\usepackage[print,1to1]{booklet}
%\source{1}{5.8in}{8.3in}
%\target{0.5}{11.7in}{8.3in}

\definecolor{ourblue}{rgb}{0,0.16,0.48}

\newcommand{\hone}[1]{{\centering \Huge\scshape #1

\bigskip

}}

\newcommand{\htwo}[1]{{\Large\scshape #1}}
\newcommand{\masspart}[1]{\bigskip

  \htwo{#1}}
\newcommand{\masssubpart}[1]{\bigskip

  {\large\scshape #1}}
\newcommand{\marriagepart}[1]{{\large\scshape #1}}
\newcommand{\note}[1]{{\centering \textit{#1}}}
\newcommand{\sayline}[2]{
  \begin{tabularx}{\textwidth}{@{} r X @{}}
    \makebox[6.0em][r]{\ifx&#1& #1 \else \textit{#1:} \fi} & #2
\end{tabularx}}
\newcommand{\mycal}[1]{\textcalligra{#1}}

\newcommand{\cparbox}[2]{\parbox{#1}{\centering #2}}
\newcommand{\partytitle}[1]{{\Large  {\mycal{#1}}}}
\newcommand{\partyname}[1]{{\footnotesize \textsc{#1}}}
\newcommand{\partyline}[2]{\partyname{#1 \linesep{} #2}}
\newcommand{\linesep}{\raisebox{1.1pt}{\fontsize{4pt}{4.8pt}\fontspec{wmdividers1.TTF}p}} % other good letters: r, s, P, y, N
\newcommand{\psep}{\bigskip{\fontspec{wmdividers1.TTF}p}\bigskip}

\newcommand{\songlisting}[2]{\textit{#1}\hdashrule{\fill}{1pt}{1pt}#2\\}
\newcommand{\reading}[1]{#1}
\newcommand{\readingquote}[1]{
\begin{quote}
``\textit{#1}''
\end{quote}
}
\newenvironment{keeptogether}{\begin{minipage}{\textwidth}}{\end{minipage}}

\begin{document}

% Shows the page layouts, with measurements and all
%\layout

%\color{ourblue}

\vspace*{\fill}

{\centering
\psep
\vspace{3em}

{\Huge
\mycal{Kathleen Mroz}\\
\mycal{\&}\\ \medskip
\mycal{Bryan Weinstein}
}

\vspace{3em}
\psep
\vspace{3em}

\textsc{
August 4, 2018\\ \medskip
2:30pm\\ \medskip
The Church of the Assumption of Our Blessed Lady\\ \medskip
Wood-Ridge, NJ}

}

\vspace*{\fill}
\newpage

% intentionally blank inside the cover
\null

\newpage

{\centering

  \hone{The Wedding Party}

  \vspace{1em}

  \begin{tabu} to \textwidth {@{} X[c] X[c] @{}}
    \noindent \partytitle{Parents of the Bride} \linebreak
    \noindent \partyname{John and Jane Smith}
  &
    \partytitle{Parents of the Groom} \linebreak
    \noindent \partyname {John and Jane Smith}
  \\ \noalign{\bigskip} % smallskip, medskip, bigskip
    \partytitle{Grandmother of the Bride} \linebreak
    \partyname{Jane Smith}
  &
    \partytitle{Grandparents of the Groom} \linebreak
    \partyname{John and Jane Smith}
  \\ \noalign{\bigskip}
  \partytitle{Maid \& Matron of Honor}\linebreak
  \partyline{Jane Smith}{Sister of the Bride}\linebreak
   \partyline{Jane Smith}{Sister of the Bride}

    \bigskip

    \partytitle{Bridesmaids}\linebreak
    \partyname{Jane Smith \linesep{} Friend of the Bride}\linebreak
    \partyname{Jane Smith \linesep{} Friend of the Bride}\linebreak
    \partyname{Jane Smith \linesep{} Friend of the Bride}\linebreak
    \partyname{Jane Smith \linesep{} Friend of the Bride}
  &
  %\centering
    \partytitle{Best Man}\linebreak \partyname{John Smith \linesep{} Cousin of the Groom}

    \bigskip

    \partytitle{Groomsmen \& Groomswoman}\linebreak
    \partyname{John Smith \linesep{} Friend of the Groom}\linebreak
    \partyname{Jane Smith \linesep{} Friend of the Groom}\linebreak
    \partyname{John Smith \linesep{} Friend of the Groom}\linebreak
    \partyname{John Smith \linesep{} Friend of the Groom}\linebreak
    \partyname{John Smith \linesep{} Brother of the Bride}
  \\ \noalign{\bigskip}
\end{tabu}

\partytitle{Celebrant}\\
\partyline{Father John Smith}{Friend of the Bride's Family}
\medskip

    \partytitle{Lectors}\\
    \partyline{Jane Smith}{Sister of the Groom}\\
    \partyline{Jane Smith}{Sister of the Groom}
    \medskip

    \partytitle{Petitioners \& Gift Bearers}\\
    \partyline{Jane Smith}{Friend of the Bride}\\
    \partyline{Jane Smith}{Friend of the Groom}
    \medskip

    \partytitle{Musician}\\
    {\footnotesize \textsc{John Smith}}

\psep

\note{It is our hope that you will join us in participating ``in full heart and voice,'' using this program as a guide for our music selections and the responses from the newest (third) edition of the Roman Missal.}

\newpage


\masspart{Riti di Introduzione}

\note{Dopo l'ingresso degli sposi, i loro genitori accendono due candele di fianco all'altare, a simboleggiare le famiglie di origine}

\masssubpart{Accoglienza}

\sayline{Celebrante}{Lorenzo e Gloria, la Chiesa partecipa alla vostra gioia e insieme con i vostri cari vi accoglie con grande affetto nel giorno in cui davanti a Dio, nostro Padre, decidete di realizzare la comunione di tutta la vita.}

\sayline{}{Per i credenti, Dio è fonte dell'amore e della fedeltà, perché Dio è amore. Ascoltiamo pertanto attentamente la sua Parola e preghiamolo con umiltà: adempia il Signore i desideri del vostro cuore ed esaudisca tutte le vostre preghiere.}


\masspart{Liturgia della Parola}

\masssubpart{Canto - Ogni mia parola (Isaia 55:10-11)}

Come la pioggia e la neve\\
scendono giù dal cielo\\
e non vi ritornano, senza irrigare\\
e far germogliare la terra,

così ogni mia parola non ritornerà a me\\
senza operare quanto desidero,\\
senza aver compiuto ciò per cui l'avevo mandata\\
ogni mia parola, ogni mia parola.

\newpage

\masssubpart{Prima lettura}

\reading{Genesi 2:18-25}
\readingquote{Questa volta è osso dalle mie ossa, carne dalla mia carne}

E il Signore Dio disse: <<Non è bene che l'uomo sia solo: voglio fargli un aiuto che gli corrisponda>>. Allora il Signore Dio plasmò dal suolo ogni sorta di animali selvatici e tutti gli uccelli del cielo e li condusse all'uomo, per vedere come li avrebbe chiamati: in qualunque modo l'uomo avesse chiamato ognuno degli esseri viventi, quello doveva essere il suo nome. Così l'uomo impose nomi a tutto il bestiame, a tutti gli uccelli del cielo e a tutti gli animali selvatici, ma per l'uomo non trovò un aiuto che gli corrispondesse. Allora il Signore Dio fece scendere un torpore sull'uomo, che si addormentò; gli tolse una delle costole e richiuse la carne al suo posto. Il Signore Dio formò con la costola, che aveva tolta all'uomo, una donna e la condusse all'uomo. Allora l'uomo disse:
\begin{quote}
<<Questa volta\\
è osso dalle mie ossa,\\
carne dalla mia carne.\\
La si chiamerà donna,\\
perché dall'uomo è stata tolta>>
\end{quote}
Per questo l'uomo lascerà suo padre e sua madre e si unirà a sua moglie, e i due saranno un'unica carne.
Ora tutti e due erano nudi, l'uomo e sua moglie, e non provavano vergogna.

\newpage

\masssubpart{Salmo (128)}

Beato chi teme il Signore\\
e cammina nelle sue vie.

Della fatica delle tue mani ti nutrirai,\\
sarai felice e avrai ogni bene.\\
La tua sposa come vite feconda\\
nell'intimità della tua casa;\\
i tuoi figli come virgulti d'ulivo\\
intorno alla tua mensa.

Ecco com'è benedetto\\
l'uomo che teme il Signore.

Ti benedica il Signore da Sion.\\
Possa tu vedere il bene di Gerusalemme\\
tutti i giorni della tua vita!\\
Possa tu vedere i figli dei tuoi figli!\\
Pace su Israele!

\masssubpart{Seconda Lettura}

\reading{Efesini 5:2a, 5:15-33}
\readingquote{Questo mistero è grande: io lo dico in riferimento a Cristo e alla Chiesa!}

[Fratelli,] camminate nella carità, nel modo in cui anche Cristo ci ha amato e ha dato sé stesso per noi.

Fate dunque molta attenzione al vostro modo di vivere, comportandovi non da stolti ma da saggi, facendo buon uso del tempo, perché i giorni sono cattivi. Non siate perciò sconsiderati, ma sappiate comprendere qual è la volontà del Signore. E non ubriacatevi di vino, che fa perdere il controllo di sé; siate invece ricolmi dello Spirito, intrattenendovi fra voi con salmi, inni, canti ispirati, cantando e inneggiando al Signore con il vostro cuore, rendendo continuamente grazie per ogni cosa a Dio Padre, nel nome del Signore nostro Gesù Cristo.

Nel timore di Cristo, siate sottomessi gli uni agli altri: le mogli lo siano ai loro mariti, come al Signore; il marito infatti è capo della moglie, così come Cristo è capo della Chiesa, lui che è salvatore del corpo. E come la Chiesa è sottomessa a Cristo, così anche le mogli lo siano ai loro mariti in tutto.

E voi, mariti, amate le vostre mogli, come anche Cristo ha amato la Chiesa e ha dato se stesso per lei, per renderla santa, purificandola con il lavacro dell'acqua mediante la parola, e per presentare a se stesso la Chiesa tutta gloriosa, senza macchia né ruga o alcunché di simile, ma santa e immacolata. Così anche i mariti hanno il dovere di amare le mogli come il proprio corpo: chi ama la propria moglie, ama se stesso. Nessuno infatti ha mai odiato la propria carne, anzi la nutre e la cura, come anche Cristo fa con la Chiesa, poiché siamo membra del suo corpo. Per questo l'uomo lascerà il padre e la madre e si unirà a sua moglie e i due diventeranno una sola carne. Questo mistero è grande: io lo dico in riferimento a Cristo e alla Chiesa! Così anche voi: ciascuno da parte sua ami la propria moglie come se stesso, e la moglie sia rispettosa verso il marito.

\masssubpart{Canto al Vangelo (in piedi)}

Alleluia, alleluia, alleluia, alleluia (2 volte)

Canto per Cristo che mi libererà\\
quando verrà nella gloria,\\
quando la vita con lui rinascerà,\\
alleluia, alleluia.

Alleluia, alleluia, alleluia, alleluia (2 volte)

\newpage

\masssubpart{Vangelo}

\reading{Giovanni 2:1-11}
\readingquote{Qualsiasi cosa vi dica, fatela.}

Il terzo giorno vi fu una festa di nozze a Cana di Galilea e c'era la madre di Gesù. Fu invitato alle nozze anche Gesù con i suoi discepoli. Venuto a mancare il vino, la madre di Gesù gli disse: <<Non hanno vino>>. E Gesù le rispose: <<Donna, che vuoi da me? Non è ancora giunta la mia ora>>. Sua madre disse ai servitori: <<Qualsiasi cosa vi dica, fatela>>.

Vi erano là sei anfore di pietra per la purificazione rituale dei Giudei, contenenti ciascuna da ottanta a centoventi litri. E Gesù disse loro: <<Riempite d'acqua le anfore>>; e le riempirono fino all'orlo. Disse loro di nuovo: <<Ora prendetene e portatene a colui che dirige il banchetto>>. Ed essi gliene portarono. Come ebbe assaggiato l'acqua diventata vino, colui che dirigeva il banchetto -- il quale non sapeva da dove venisse, ma lo sapevano i servitori che avevano preso l'acqua -- chiamò lo sposo e gli disse: <<Tutti mettono in tavola il vino buono all'inizio e, quando si è già bevuto molto, quello meno buono. Tu invece hai tenuto da parte il vino buono finora>>.

Questo, a Cana di Galilea, fu l'inizio dei segni compiuti da Gesù; egli manifestò la sua gloria e i suoi discepoli credettero in lui.


\masssubpart{Omelia (seduti)}

\newpage

\masspart{Rito del Matrimonio}

\masssubpart{Manifestazione del consenso e scambio dei voti}

\sayline{Celebrante}{Carissimi, siete qui convenuti davanti ai ministri della Chiesa e davanti alla comunità perché la vostra decisione di unirvi in Matrimonio sia fortificata dal sigillo del Signore e il vostro amore, arricchito della Sua benedizione, sia rafforzato nella reciproca e perpetua fedeltà e nel compimento degli altri doveri del Matrimonio.}

\sayline{}{Vi chiedo pertanto di esprimere davanti alla Chiesa le vostre intenzioni.}

\sayline{}{Lorenzo e Gloria, siete venuti a celebrare il Matrimonio senza alcuna costrizione, in piena libertà e consapevoli del significato della vostra decisione?}

\sayline{Sposi}{Sì}

\sayline{Celebrante:}{Siete disposti, seguendo la via del Matrimonio, ad amarvi e a onorarvi l'un l'altro per tutta la vita?}

\sayline{Sposi}{Sì}

\sayline{Celebrante}{Siete disposti ad accogliere con amore i figli che Dio vorrà donarvi e a educarli secondo la legge di Cristo e della sua Chiesa?}

\sayline{Sposi}{Sì}

\sayline{Celebrante}{Se è vostra intenzione di unirvi in matrimonio, datevi la mano destra ed esprimete davanti a Dio e alla sua Chiesa il vostro consenso.}

\newpage

\note{Gli sposi si danno la mano destra.}

\sayline{Lorenzo}{Io, Lorenzo, accolgo te, Gloria, come mia sposa. Prometto di esserti fedele sempre, nella gioia e nel dolore, nella salute e nella malattia, e di amarti e onorarti tutti i giorni della mia vita.}

\sayline{Gloria}{Io, Gloria, accolgo te, Lorenzo, come mio sposo. Prometto di esserti fedele sempre, nella gioia e nel dolore, nella salute e nella malattia, e di amarti e onorarti tutti i giorni della mia vita.}

\masssubpart{Benedizione e consegna degli anelli}

\sayline{Celebrante}{Il Signore benedica questi anelli che vi donate scambievolmente in segno di amore e di fedeltà.}

\sayline{Lorenzo}{Gloria, ricevi questo anello, segno del mio amore e della mia fedeltà, nel nome del Padre, del Figlio e dello Spirito Santo.}

\sayline{Gloria}{Lorenzo, ricevi questo anello, segno del mio amore e della mia fedeltà, nel nome del Padre, del Figlio e dello Spirito Santo.}

\newpage

\masssubpart{Benedizione nuziale}

\sayline{Celebrante}{Invochiamo ora su questi sposi la benedizione di Dio: egli sostenga con il Suo aiuto coloro che ha arricchito con la comunione di vita del Matrimonio.}

\sayline{}{Padre santo, creatore dell'universo,\\%
&che hai formato l'uomo e la donna a tua immagine\\%
&e hai voluto benedire la loro unione,\\%
&ti preghiamo umilmente per questi tuoi figli,\\%
&che oggi si uniscono con il patto nuziale.}

\sayline{}{Scenda su questi sposi, Lorenzo e Gloria,\\
&la ricchezza delle tue benedizioni\\
&e la forza del tuo Santo Spirito infiammi i loro cuori,\\
&perché, mentre vivono il reciproco dono di amore,\\
&siano esemplari per integrità di vita\\
&e genitori saldi nella virtù.}

\sayline{}{Ti lodino, Signore, nella gioia,\\
&ti cerchino nella sofferenza;\\
&godano del tuo sostegno nella fatica\\
&e del tuo conforto nella necessità.}

\sayline{}{Vivano a lungo nella prosperità e nella pace\\
&e, con tutti gli amici che ora li circondano,\\
&giungano alla felicità del tuo regno.\\
&Per Cristo nostro Signore.}

\sayline{Tutti:}{Amen}

\newpage

\masssubpart{Canto - Il Cielo Canta}

\note{Gli sposi accendono la terza candela accanto all'altare, a simboleggiare la nascita di una nuova famiglia}

Canta il cielo\\
di gioia, alleluia!\\
Quando la gloria di Dio\\
splende su te e su me.\\
Alleluia, alleluia!\\
Alleluia, alleluia!

Canta il cielo\\
di gioia, alleluia!\\
Quando la gloria di Dio\\
ci unisce insiem\\
Alleluia, alleluia!\\
Alleuia, alleluia!

\masssubpart{Preghiera dei fedeli}

\sayline{Celebrante}{Invochiamo Dio, nostro Padre, sorgente inesauribile dell'amore, perché sostenga questi sposi nel cammino che oggi hanno iniziato.}

\sayline{Lettore}{Per Lorenzo e Gloria: il Signore li sostenga nella donazione reciproca, e renda la loro unione feconda e piena di grazia e gioia. Preghiamo.}

\sayline{Tutti}{Ascoltaci, o Padre}

\sayline{Lettore}{Per Lorenzo e Gloria: la grazia della benedizione del Signore li conforti nelle difficoltà e li custodisca nella fedeltà. Preghiamo.}

\sayline{Tutti}{Ascoltaci, o Padre}

\newpage

\sayline{Lettore}{Per le famiglie di Lorenzo e Gloria, perché li sostengano con la loro preghiera, il loro affetto e il loro consiglio. Preghiamo.}

\sayline{Tutti}{Ascoltaci, o Padre}

\sayline{Lettore}{Per i fidanzati: riconoscenti per il dono del loro amore, si preparino a costruire la loro famiglia secondo la Parola del Vangelo. Preghiamo.}

\sayline{Tutti}{Ascoltaci, o Padre}

\sayline{Lettore}{Per gli sposi cristiani qui presenti: possano attingere da questa celebrazione luce e forza, per rinnovare la grazia del loro matrimonio.}

\sayline{Tutti}{Ascoltaci, o Padre}

\sayline{Lettore}{Per il popolo cristiano: cresca ogni giorno nell'unità e nella fede, e sappia esprimere nella carità reciproca il suo rapporto d'amore con Dio.}

\sayline{Tutti}{Ascoltaci, o Padre}

\note{Chi vuole, può aggiungere una preghiera spontanea}

\sayline{Celebrante}{Dio vuole che tutti i suoi figli siano concordi nell'amore. Coloro che credono in Cristo invochino il Padre con la preghiera della famiglia di Dio, che il Signore Gesù ci ha insegnato.}

\sayline{Tutti}{Padre nostro, che sei nei cieli\\
&sia santificato il tuo nome\\
&venga il tuo regno, sia fatta la tua volontà\\
&come in cielo così in terra.}

\sayline{}{Dacci oggi il nostro pane quotidiano\\
&e rimetti a noi i nostri debiti,\\
&come noi li rimettiamo ai nostri debitori,\\
&e non ci indurre in tentazione, ma liberaci dal male.}

\sayline{}{Amen.}

\masspart{Riti di conclusione}

\masssubpart{Lettura degli articoli del Codice Civile}

\note{Il celebrante legge gli articoli di legge riguardanti il matrimonio}

\masssubpart{Benedizione finale}

\note{Viene data la benedizione finale. Ad ogni invocazione si risponde ``Amen''}

\masssubpart{Firma dell'atto di matrimonio}

\note{Viene data lettura dell'atto di matrimonio, che viene firmato da sposi e testimoni}

\newpage

\masssubpart{Canto - Canto del Mare (Esodo 15)}

Cantiamo al Signore,\\
stupenda è la Sua vittoria.\\
Signore è il suo nome,\\
Alleluia. (x2)

Voglio cantare in onore del Signore\\
perché ha trionfato, alleluia.\\
Ha gettato in mare cavallo e cavaliere.\\
Mia forza e mio canto è il Signore,\\
il mio Salvatore è il Dio di mio padre\\
ed io lo voglio esaltare.

Cantiamo al Signore,\\
stupenda è la Sua vittoria.\\
Signore è il suo nome,\\
Alleluia. (x2)

Si accumularon le acque al suo soffio\\
s'alzarono le onde come un argine.\\
Si raggelaron gli abissi in fondo al mare.\\
Chi è come te, o Signore?\\
Guidasti con forza il popolo redento\\
e lo conducesti verso Sion.


\newpage
\null

\newpage
\null

\end{document}
